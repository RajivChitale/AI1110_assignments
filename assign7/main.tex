\let\negmedspace\undefined
\let\negthickspace\undefined
%\RequirePackage{amsmath}
% Inbuilt themes in beamer
\documentclass{beamer}
% Theme choice:
\usetheme{CambridgeUS}

\usepackage{gensymb}
\usepackage{polynom}
\usepackage{amssymb}
\usepackage{amsthm}
\usepackage{stfloats}
\usepackage{bm}
%\usepackage{enumitem}
\usepackage{mathtools}
  %\usepackage{tikz}
% \usepackage{circuitikz}
% \usepackage{verbatim}
%\usepackage{tfrupee}
  %\usepackage[breaklinks=true]{hyperref}
%\usepackage{stmaryrd}
%\usepackage{tkz-euclide} % loads  TikZ and tkz-base
%\usetkzobj{all}
\usepackage{listings}
    \usepackage{color}                                            
    \usepackage{array}                                            
    \usepackage{longtable}                                        
    \usepackage{calc}                                             
    \usepackage{multirow}                                         
    \usepackage{hhline}                                           
    \usepackage{ifthen}                                           
  %optionally (for landscape tables embedded in another document): 
    \usepackage{lscape}     
% \usepackage{multicol}
% \usepackage{chngcntr}
%\usepackage{enumerate}
%\usepackage{wasysym}
%\newcounter{MYtempeqncnt}
\DeclareMathOperator*{\Res}{Res}
\DeclareMathOperator*{\equals}{=}
%\renewcommand{\baselinestretch}{2}
\renewcommand\thesection{\arabic{section}}
\renewcommand\thesubsection{\thesection.\arabic{subsection}}
\renewcommand\thesubsubsection{\thesubsection.\arabic{subsubsection}}

% correct bad hyphenation here
\hyphenation{op-tical net-works semi-conduc-tor}
\def\inputGnumericTable{}                                 
\lstset{
%language=C,
frame=single, 
breaklines=true,
columns=fullflexible
}
%
\begin{document}
%
\newcommand{\BEQA}{\begin{eqnarray}}
\newcommand{\EEQA}{\end{eqnarray}}
\newcommand{\define}{\stackrel{\triangle}{=}}
\newcommand*\circled[1]{\tikz[baseline=(char.base)]{
    \node[shape=circle,draw,inner sep=2pt] (char) {#1};}}
\bibliographystyle{IEEEtran}
%\bibliographystyle{ieeetr}
%
\providecommand{\mbf}{\mathbf}
\providecommand{\pr}[1]{\ensuremath{\Pr\left(#1\right)}}
\providecommand{\qfunc}[1]{\ensuremath{Q\left(#1\right)}}
\providecommand{\sbrak}[1]{\ensuremath{{}\left[#1\right]}}      % []
\providecommand{\lsbrak}[1]{\ensuremath{{}\left[#1\right.}}
\providecommand{\rsbrak}[1]{\ensuremath{{}\left.#1\right]}}
\providecommand{\brak}[1]{\ensuremath{\left(#1\right)}}         % ()
\providecommand{\lbrak}[1]{\ensuremath{\left(#1\right.}}
\providecommand{\rbrak}[1]{\ensuremath{\left.#1\right)}}
\providecommand{\cbrak}[1]{\ensuremath{\left\{#1\right\}}}      % {}
\providecommand{\lcbrak}[1]{\ensuremath{\left\{#1\right.}}
\providecommand{\rcbrak}[1]{\ensuremath{\left.#1\right\}}}
\theoremstyle{remark}
\newtheorem{rem}{Remark}
\newcommand{\sgn}{\mathop{\mathrm{sgn}}}
\providecommand{\abs}[1]{\ensuremath{\left\vert#1\right\vert}}
\providecommand{\res}[1]{\Res\displaylimits_{#1}} 
\providecommand{\norm}[1]{\ensuremath{\left\lVert#1\right\rVert}}
%\providecommand{\norm}[1]{\lVert#1\rVert}
\providecommand{\mtx}[1]{\mathbf{#1}}
\providecommand{\mean}[1]{\ensuremath{E\left[ #1 \right]}}
\providecommand{\fourier}{\overset{\mathcal{F}}{ \rightleftharpoons}}
%\providecommand{\hilbert}{\overset{\mathcal{H}}{ \rightleftharpoons}}
\providecommand{\system}{\overset{\mathcal{H}}{ \longleftrightarrow}}
	%\newcommand{\solution}[2]{\textbf{Solution:}{#1}}
\newcommand{\cosec}{\,\text{cosec}\,}
\providecommand{\dec}[2]{\ensuremath{\overset{#1}{\underset{#2}{\gtrless}}}}
\newcommand{\myvec}[1]{\ensuremath{\begin{pmatrix}#1\end{pmatrix}}}
\newcommand{\mydet}[1]{\ensuremath{\begin{vmatrix}#1\end{vmatrix}}}
\newcommand*{\permcomb}[4][0mu]{{{}^{#3}\mkern#1#2_{#4}}}
\newcommand*{\perm}[1][-3mu]{\permcomb[#1]{P}}
\newcommand*{\comb}[1][-1mu]{\permcomb[#1]{C}}
%
%not used because document is short:
%\numberwithin{equation}{section}
%\numberwithin{figure}{section}
%\numberwithin{table}{section}
%\numberwithin{equation}{section}
%\numberwithin{problem}{section}
%\numberwithin{definition}{section}
\makeatletter
\@addtoreset{figure}{problem}
\makeatother

\let\StandardTheFigure\thefigure
\let\vec\mathbf
%\renewcommand{\thefigure}{\theproblem.\arabic{figure}}
    %\renewcommand{\thefigure}{\theproblem}
%\setlist[enumerate,1]{before=\renewcommand\theequation{\theenumi.\arabic{equation}}
%\counterwithin{equation}{enumi}
%\renewcommand{\theequation}{\arabic{subsection}.\arabic{equation}}

\def\putbox#1#2#3{\makebox[0in][l]{\makebox[#1][l]{}\raisebox{\baselineskip}[0in][0in]{\raisebox{#2}[0in][0in]{#3}}}}
     \def\rightbox#1{\makebox[0in][r]{#1}}
     \def\centbox#1{\makebox[0in]{#1}}
     \def\topbox#1{\raisebox{-\baselineskip}[0in][0in]{#1}}
     \def\midbox#1{\raisebox{-0.5\baselineskip}[0in][0in]{#1}}
\vspace{3cm}

%\renewcommand{\thefigure}{\theenumi}
%\renewcommand{\thetable}{\theenumi}
%\renewcommand{\theequation}{\theenumi}

% Title page details: 
\title{Assignment 7, AI1110} 
\author{Rajiv Shailesh Chitale (cs21btech11051)}
\date{\today}
\logo{\large \LaTeX{}}
% Title page frame
\begin{frame}
    \titlepage 
\end{frame}
%
\logo{}
%
\begin{frame}{Example 7.3 Papoulis}
    In this example, the following are discussed
    \tableofcontents
\end{frame}
%
\section{Probability of Components Being Good}
\begin{frame}{Probability of Components Being Good}
    \begin{itemize}
        \item A system consists of $m$ components. The time to failure of the $i^{th}$ component is a random variable $X_i$.
        \item Its cumulative distribution $F_i\brak{X_i}$ denotes the probability that $i^{th}$ component has fails at or before time $t$. 
        \item The probability that the $i^{th}$ component is good at time $t$ is 
        \begin{align}
            1 - F_i\brak{t} = P\brak{X_i > t} 
        \end{align}
    \end{itemize}
\end{frame}
%
\section{Number of Good Components}
\begin{frame}{Number of Good Components}
    Let $n\brak{t}$ denote the number
    of components that are good at time t. Then,
    \begin{align}
        n\brak{t} &= n_1 + \dots + n_m \\
        n_i &=
        \begin{cases} 
        1 & X_i > t \\
        0 & X_i < t
        \end{cases}
    \end{align} 
\end{frame}
%
\section{Expected Number of Good Components}
\begin{frame}{Expected Number of Good Components}
\begin{itemize}
    \item The expectation value of a component being good at time $t$ is
    \begin{align}
        E\cbrak{n_i} &= 0 \times p\brak{n_i = 0}
            + 1 \times p\brak{n_i = 1} \\
        E\cbrak{n_i} &= 0 + 1 \times P\brak{X_i > t} \\
        E\cbrak{n_i} &= 1 - F_i\brak{t}
    \end{align}
    \item We obtain the expectation value of $n\brak{t}$,
    \begin{align}
      \eta\brak{t} &= E\cbrak{n\brak{t}} \\
      \eta\brak{t} &= 1 - F_1\brak{t} + \dots + 1 - F_m\brak{t} 
    \end{align}
    \item If we assume the case that each $X_i$ has the same distribution $F\brak{t}$, then
    \begin{align}
    \eta\brak{t} &= m\sbrak{1 - F\brak{t}}
    \label{eq:meanNumber}
    \end{align}
\end{itemize}
\end{frame}
%
\section{Failure Rate}
\begin{frame}{Failure Rate}
    \begin{itemize}
    \item The difference $\eta\brak{t} - \eta\brak{t+dt} $ is the expected number of failures in the interval $\brak{t, t+dt}$.
    \item The rate of failure is given by $- \eta'\brak{t}$
    \begin{align}
       -\eta'\brak{t} &= \frac{\eta\brak{t} - \eta\brak{t+dt}}{dt}  
    \end{align}
    \item Equation \eqref{eq:meanNumber} can be differentiated to obtain,
    \begin{align}
       -\eta'\brak{t} &= m \times f\brak{t}
    \end{align}
    \end{itemize}
\end{frame}
%
\section{Relative Expected Failure Rate}
\begin{frame}{Relative Expected Failure Rate}
    The relative expected failure rate is calculated with respect to the number of components that are good at time $t$. It is given by the ratio,
    \begin{align}
        \beta\brak{t} &= \frac{-\eta'\brak{t}} {\eta\brak{t}} 
        \label{eq:beta} \\
        \beta\brak{t} &= \frac{f\brak{t}}{1- F\brak{t}} 
    \end{align}
    On integrating \eqref{eq:beta} we obtain,
    \begin{align}
        - \int_0^t \beta\brak{\tau} d\tau = \ln{\eta\brak{t}} - \ln{\eta\brak{0}} 
    \end{align}
\end{frame}
%
\begin{frame}{}
    \begin{itemize}
        \item Let us assume operations start at $t=0$ with $n\brak{0}=m$.
        \item Then we have $ \eta\brak{0} = E\cbrak{n\brak{0}} = m$
    \end{itemize}
    \begin{align}
        - \int_0^t \beta\brak{\tau} d\tau &= \ln{\eta\brak{t}} - \ln{m} \\
     \implies  \eta\brak{t} &= m \exp{\cbrak{- \int_0^t \beta\brak{\tau} d\tau} }
    \end{align}
\end{frame}
\end{document}

