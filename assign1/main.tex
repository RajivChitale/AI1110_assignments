\documentclass[journal,12pt,twocolumn]{IEEEtran}
\usepackage[utf8]{inputenc}
\usepackage{amsmath}

\newcommand{\myvec}[1]{\ensuremath{\begin{pmatrix}#1\end{pmatrix}}}
\let\vec\mathbf
%to write vector and matrix names in bold

\title{Assignment 1, AI1110}
\author{Rajiv Shailesh Chitale (cs21btech11051)}
\date{2 April 2022}

\pagenumbering{gobble}
\begin{document}
\maketitle

\subsection*{\textbf{ICSE 10th, 2017 Paper}}
\subsection*{\textbf{Question 5(a)}}
    Given matrix $\vec{B} = \myvec{ 1 & 1 \\8 & 3}$
    . Find the matrix 
    \\
    \textbf{X} if,  $\vec{X} = \vec{B}^2 -4\vec{B}$. Hence solve for a and b
    \\
    given $\vec{X} \myvec{a \\ b}
    = \myvec{5\\50} $
\subsection*{\textbf{Solution:}}
    First, obtain the characteristic equation of B,
    \begin{align}
        |\vec{B}-\lambda \vec{I}| &= 0 
        \\
        \begin{vmatrix} 1-\lambda & 1\\ 8 & 3-\lambda \end{vmatrix} &= 0 
        \\
        (1-\lambda)(3-\lambda)-(8)(1) &= 0
        \\
        \lambda ^2 - 4\lambda -5 &= 0
    \end{align}
        From Cayley-Hamilton theorem,
        \begin{align} 
        \vec{B}^2 -4\vec{B} - 5\vec{I} &= 0
        \\
        \vec{B}^2 -4\vec{B} &= 5\vec{I}
        \\
        \Rightarrow \vec{X} &= 5\vec{I}
    \end{align}
    Thus we obtain $\vec{X} \myvec{a \\ b}
    = \myvec{5a \\ 5b} $
    \hfill (8)
    \\
    \\
    It is given that $\vec{X} \myvec{a \\ b}
    = \myvec{5 \\ 50} $
    \hfill (9)
    \\
    From (8) and (9),
    \[ \Rightarrow \myvec{5a \\ 5b}
    =\myvec{5 \\ 50} \]
    On equating elements $a = 1 , b = 10$
    
\end{document}

