\let\negmedspace\undefined
\let\negthickspace\undefined
%\RequirePackage{amsmath}
% Inbuilt themes in beamer
\documentclass{beamer}
% Theme choice:
\usetheme{CambridgeUS}

\usepackage{gensymb}
\usepackage{polynom}
\usepackage{amssymb}
\usepackage{amsthm}
\usepackage{stfloats}
\usepackage{bm}
\usepackage{enumitem}
\usepackage{mathtools}
  \usepackage{tikz}
% \usepackage{circuitikz}
% \usepackage{verbatim}
%\usepackage{tfrupee}
  %\usepackage[breaklinks=true]{hyperref}
%\usepackage{stmaryrd}
%\usepackage{tkz-euclide} % loads  TikZ and tkz-base
%\usetkzobj{all}
\usepackage{listings}
    \usepackage{color}                                            
    \usepackage{array}                                            
    \usepackage{longtable}                                        
    \usepackage{calc}                                             
    \usepackage{multirow}                                         
    \usepackage{hhline}                                           
    \usepackage{ifthen}                                           
  %optionally (for landscape tables embedded in another document): 
    \usepackage{lscape}     
% \usepackage{multicol}
% \usepackage{chngcntr}
%\usepackage{enumerate}
%\usepackage{wasysym}
%\newcounter{MYtempeqncnt}
\DeclareMathOperator*{\Res}{Res}
\DeclareMathOperator*{\equals}{=}
%\renewcommand{\baselinestretch}{2}
\renewcommand\thesection{\arabic{section}}
\renewcommand\thesubsection{\thesection.\arabic{subsection}}
\renewcommand\thesubsubsection{\thesubsection.\arabic{subsubsection}}

% correct bad hyphenation here
\hyphenation{op-tical net-works semi-conduc-tor}
\def\inputGnumericTable{}                                 
\lstset{
%language=C,
frame=single, 
breaklines=true,
columns=fullflexible
}
%
\begin{document}
%
\newcommand{\BEQA}{\begin{eqnarray}}
\newcommand{\EEQA}{\end{eqnarray}}
\newcommand{\define}{\stackrel{\triangle}{=}}
\newcommand*\circled[1]{\tikz[baseline=(char.base)]{
    \node[shape=circle,draw,inner sep=2pt] (char) {#1};}}
\bibliographystyle{IEEEtran}
%\bibliographystyle{ieeetr}
%
\providecommand{\mbf}{\mathbf}
\providecommand{\pr}[1]{\ensuremath{\Pr\left(#1\right)}}
\providecommand{\qfunc}[1]{\ensuremath{Q\left(#1\right)}}
\providecommand{\sbrak}[1]{\ensuremath{{}\left[#1\right]}}      % []
\providecommand{\lsbrak}[1]{\ensuremath{{}\left[#1\right.}}
\providecommand{\rsbrak}[1]{\ensuremath{{}\left.#1\right]}}
\providecommand{\brak}[1]{\ensuremath{\left(#1\right)}}         % ()
\providecommand{\lbrak}[1]{\ensuremath{\left(#1\right.}}
\providecommand{\rbrak}[1]{\ensuremath{\left.#1\right)}}
\providecommand{\cbrak}[1]{\ensuremath{\left\{#1\right\}}}      % {}
\providecommand{\lcbrak}[1]{\ensuremath{\left\{#1\right.}}
\providecommand{\rcbrak}[1]{\ensuremath{\left.#1\right\}}}
\theoremstyle{remark}
\newtheorem{rem}{Remark}
\newcommand{\sgn}{\mathop{\mathrm{sgn}}}
\providecommand{\abs}[1]{\ensuremath{\left\vert#1\right\vert}}
\providecommand{\res}[1]{\Res\displaylimits_{#1}} 
\providecommand{\norm}[1]{\ensuremath{\left\lVert#1\right\rVert}}
%\providecommand{\norm}[1]{\lVert#1\rVert}
\providecommand{\mtx}[1]{\mathbf{#1}}
\providecommand{\mean}[1]{\ensuremath{E\left[ #1 \right]}}
\providecommand{\fourier}{\overset{\mathcal{F}}{ \rightleftharpoons}}
%\providecommand{\hilbert}{\overset{\mathcal{H}}{ \rightleftharpoons}}
\providecommand{\system}{\overset{\mathcal{H}}{ \longleftrightarrow}}
	%\newcommand{\solution}[2]{\textbf{Solution:}{#1}}
\newcommand{\cosec}{\,\text{cosec}\,}
\providecommand{\dec}[2]{\ensuremath{\overset{#1}{\underset{#2}{\gtrless}}}}
\newcommand{\myvec}[1]{\ensuremath{\begin{pmatrix}#1\end{pmatrix}}}
\newcommand{\mydet}[1]{\ensuremath{\begin{vmatrix}#1\end{vmatrix}}}
\newcommand*{\permcomb}[4][0mu]{{{}^{#3}\mkern#1#2_{#4}}}
\newcommand*{\perm}[1][-3mu]{\permcomb[#1]{P}}
\newcommand*{\comb}[1][-1mu]{\permcomb[#1]{C}}
%
%not used because document is short:
%\numberwithin{equation}{section}
%\numberwithin{figure}{section}
%\numberwithin{table}{section}
%\numberwithin{equation}{section}
%\numberwithin{problem}{section}
%\numberwithin{definition}{section}
\makeatletter
\@addtoreset{figure}{problem}
\makeatother

\let\StandardTheFigure\thefigure
\let\vec\mathbf
%\renewcommand{\thefigure}{\theproblem.\arabic{figure}}
    %\renewcommand{\thefigure}{\theproblem}
%\setlist[enumerate,1]{before=\renewcommand\theequation{\theenumi.\arabic{equation}}
%\counterwithin{equation}{enumi}
%\renewcommand{\theequation}{\arabic{subsection}.\arabic{equation}}

\def\putbox#1#2#3{\makebox[0in][l]{\makebox[#1][l]{}\raisebox{\baselineskip}[0in][0in]{\raisebox{#2}[0in][0in]{#3}}}}
     \def\rightbox#1{\makebox[0in][r]{#1}}
     \def\centbox#1{\makebox[0in]{#1}}
     \def\topbox#1{\raisebox{-\baselineskip}[0in][0in]{#1}}
     \def\midbox#1{\raisebox{-0.5\baselineskip}[0in][0in]{#1}}
\vspace{3cm}

%\renewcommand{\thefigure}{\theenumi}
%\renewcommand{\thetable}{\theenumi}
%\renewcommand{\theequation}{\theenumi}

% Title page details: 
\title{Assignment 6, AI1110} 
\author{Rajiv Shailesh Chitale (cs21btech11051)}
\date{\today}
\logo{\large \LaTeX{}}
% Title page frame
\begin{frame}
    \titlepage 
\end{frame}

\logo{}

% Outline frame
\begin{frame}{Outline}
    \tableofcontents
\end{frame}
\section{Question}

\begin{frame}{Question}
    %\textbf{Question 15:}
     \begin{block}{ Question 15, NCERT class 12 Probability Ex 13.1}
    Consider the experiment of throwing a die.
    \begin{itemize}
        \item If a multiple of 3 comes up, throw the die again
        \item If any other number comes, toss a coin.
    \end{itemize}
     Find the conditional probability of the event \lq the coin shows a tail\rq, given that \lq at least one die shows a 3\rq.
     \end{block} 
\end{frame}
%%
\section{Markov Chain States}
\begin{frame}{Markov Chain}
    Let us construct a Markov chain $X_t$ with discrete time t. The states $e_1$, $e_2$ and $e_3$ describe the outcomes from the latest dice throw. The states $e_4$ and $e_5$ describe the outcomes of the latest coin toss.
\end{frame}
%%
\begin{frame}{States}
    Let $ Y \in \cbrak{1,2,3,4,5,6} $ denote the number obtained from a die throw. 
    \begin{table}[ht!]
        \centering
    	\input{tables/table1.tex}
        \caption{States in Markov Chain}
        \label{table:States}	
    \end{table}
\end{frame}
%%
\section{Markov Chain Graph}
\begin{frame}{Graph of Markov Chain}
    \begin{figure}[!ht]
        \centering
        \begin{tikzpicture}[->, >= stealth, shorten >=2pt , line width =0.5 pt ,
            node distance =2 cm]
            \input{graph/graph1.txt}  
        \end{tikzpicture}
        \caption{Graph of Markov Chain}
        \label{fig: markov_chain}
    \end{figure}
\end{frame}
%%
\section{Transition Probability Matrix}
\begin{frame}{Transition Probability Matrix}
    $p_{j/i}$ is the probability of moving from state $e_i$ to $e_j$. 
    \begin{align}
    p_{j/i} = \pr{\frac{ X_{t+1}=e_j } {X_t=e_i} }
    \end{align}
    
    These probabilities are contained in the transition probability matrix.
    \begin{align}
        \vec{P_{ij}}= \myvec{\input{tables/table2.txt}} 
    \end{align}
\end{frame}
%%
\section{Limiting State Vector}
\begin{frame}{Limiting state vector}
    $\vec{Q_t}$ is the state vector at a given t. The given condition is that 3 occurs at least once. Let the first occurrence of 3 be the initial state $\vec{Q_0}$.
    \begin{align}
        \vec{Q_0} &= \myvec{\input{tables/table3.txt}} \\
        \vec{Q_t} &= \vec{Q_0} \vec{P}^{t}  
    \end{align}
    %  
    The limiting probabilities of states, calculated from solve.py,
    \begin{align}
       \lim_{t \to \infty}  \vec{Q_t} &= 
       \myvec{\input{tables/table4.txt}} 
    \end{align}
    %
    Required conditional probability is,
    \begin{align}
       \lim_{t \to \infty} \pr{X_t=e_4} = 0.5
    \end{align}
\end{frame}
%%
\end{document}
